\section{BPE: Business Processes}
The author of BPE has implemented business workflow engine for CLR virtual
machine using C\# language. However Erlang implementation more idiomatic
and canonical due to semantic corresponding of process calculus and
the core of underlying virtual machine. Send async messages across processes
means exactly what it says up to Erlang pids. For sending documents to
business process you can use process's name or its Pid:

\vspace{1\baselineskip}
\begin{lstlisting}
    1> bpe:amend(pid(0,185,0),#deposit{}).
\end{lstlisting}

\paragraph{}
Thanks to this isomorphic corresponding between Erlang process and Calculus process,
code size of core BPE server was reduced to 400 lines of code. This is definitely
most clean functional implementation of workflow engine.

\subsection{Pi-calculus and Petri nets}
The nice thing about all palette of different implementation of workflow models is
that all of them reduced to one of two kinds of encoding: one is algebraic one and
the other is geometric.

\paragraph{}
The geometric one is {\bf Petri nets}. Carl Petri introduced it in 1962 during discrete
analysis of asynchronous computer systems. Any its graphical representation could be
defined with Petri nets formalism. Petri modeling in one of its forms is a good
complementation to process algebra useful as computational model.

\paragraph{}
The algebraic one is {\bf Pi-calculus} developed by Robin Milner who gained
Turing award for 1) Meta Language ML, 2) Calculus for Communication Systems CCS (1980),
the general theory of concurrency and 3) theoretical base for proof assistants,
Logic for Computable Functions LCF.
The model of process calculus is a theoretical
background of virtual environment of Erlang infrastructure, so BPE
implementation fully relies on Pi-calculus (1999), the successor of CCS notion.
Thus providing effective computational model for implementation of workflow
process management.

\subsection{Finite State Machines}
One of the common known types of encoding process calculus is well developed {\bf FSM framework} (60-s).
This language is widely used almost in any programming language presented as core
feature or as library. The process defines with an extension to Turing machine
with states, input, outputs and functions.

\subsection{SADT}
The next language (framework) that used in (80-s, 90-s) to describe similar
to process calculus definitions with graphical Petri nets and model definitions
was {\bf SADT} introduced by Marca and MacGowan 1988, 1991.

\subsection{Reactive Systems}
One of the wide range of semantics is Reactive Systems based on message passing
and event routing, but also it could be known as Functional Reactive Programming FRP
which is rather a set of combinators over streams. Both interpretations are
used in languages and frameworks, depending on involvement of stream in core
definition (2010-s).

\subsection{Typing Pi-calculus}
In typed theory Pi-calculus defines also the typing system (could be System F, e.g.) for
input and outputs of processes or function signatures specified in process definition.
In BPE the role of types was taken by document types, which is simple Erlang records,
so in BPE workflow processing is type-safe on compilation stage with respect to document types.

\vspace{1\baselineskip}
\begin{lstlisting}
    1> #deposit{} = bpe:doc(#deposit{}, pid(0,185,0)).
\end{lstlisting}
\subsection{Systems}

\subsection{Scenarios}
Workflows are complimentary to business rules and could be specified separately.
BPE definitions provides front API to the end-user application.
Workflow Engine -- is an Erlang/OTP application which handles process definitions,
process instances, and provides very clean API for Workplaces.

\paragraph{}
Before using Process Engine you need to define the set of business process
workflows of your enterprise. This could be done via Erlang terms or some DSL
that lately converted to Erlang terms. Internally BPE uses Erlang terms
workflow definition:

\vspace{1\baselineskip}
\begin{lstlisting}
bpe:start(
   #process{name="Order Form",
       flows=[ #sequenceFlow{source="begin",target="fill_form"},
               #sequenceFlow{source="fill_form",target="end"}],
       tasks=[ #userTask{name="begin"},
               #userTask{name="fill_form"},
               #endEvent{name="end"}],
       task="begin",beginEvent="begin",endEvent="end"},[]).
\end{lstlisting}


\newpage
Slightly bigger example:

\vspace{1\baselineskip}
\begin{lstlisting}
deposit_app() ->

    #process { name = 'Create Deposit Account',

        flows = [
            #sequenceFlow{source='Init',      target='Payment'},
            #sequenceFlow{source='Payment',   target='Signatory'},
            #sequenceFlow{source='Payment',   target='Process'},
            #sequenceFlow{source='Process',   target='Final'},
            #sequenceFlow{source='Signatory', target='Process'},
            #sequenceFlow{source='Signatory', target='Final'}
        ],

        tasks = [
            #userTask    { name='Init',      module = deposit },
            #userTask    { name='Signatory', module = deposit},
            #serviceTask { name='Payment',   module = deposit},
            #serviceTask { name='Process',   module = deposit},
            #endEvent    { name='Final'}
        ],

        beginEvent = 'Init',
        endEvent = 'Final',
        events = [
             #messageEvent{name="PaymentReceived"}
        ]
    }.
\end{lstlisting}

\subsection{Actions}
Business rules should be specified by developers.
RETE is used for rules specifications which can be triggered during workflow.

\newpage

\subsection{BPMN 2.0}

The workflow definition uses following persistent workflow model which is stored in KVS:


\vspace{1\baselineskip}
\begin{lstlisting}
-record(task,         { name, id, roles, module }).
-record(userTask,     { name, id, roles, module }).
-record(serviceTask,  { name, id, roles, module }).
-record(messageEvent, { name, id, payload }).
-record(beginEvent ,  { name, id }).
-record(endEvent,     { name, id }).
-record(sequenceFlow, { name, id, source, target }).
-record(history,      { ?ITERATOR(feed,true), name, task }).
-record(process,      { ?ITERATOR(feed,true), name,
                        roles=[], tasks=[], events=[],
                        history=[], flows=[], rules,
                        docs=[], task, beginEvent,
                        endEvent }).
\end{lstlisting}

Full set of BPMN 2.0 fields could be obtained at \footahref{http://www.omg.org/spec/BPMN/2.0}{OMG definition document, page 3-7}.

\newpage

\subsection{Sample Process}

\subsection{Erlang Session}

\vspace{1\baselineskip}
\begin{lstlisting}
(bpe@127.0.0.1)1> kvs:join().
ok

(bpe@127.0.0.1)1> rr(bpe).
[beginEvent,container,endEvent,history,id_seq,iterator,
 messageEvent,process,sequenceFlow,serviceTask,task,userTask]

(bpe@127.0.0.1)2> bpe:start(#process{name="Order11",
         flows=[#sequenceFlow{source="begin",target="end2"},
                #sequenceFlow{source="end2",target="end"}],
         tasks=[#userTask{name="begin"},
                #userTask{name="end2"},
                #endEvent{name="end"}],
         task="begin",beginEvent="begin",endEvent="end"},[]).
bpe_proc:Process 39 spawned <0.12399.0>
{ok,<0.12399.0>}

(bpe@127.0.0.1)3> gen_server:call(pid(0,12399,0),{complete}).

(bpe@127.0.0.1)4> gen_server:call(pid(0,12399,0),{complete}).

(bpe@127.0.0.1)5> gen_server:call(pid(0,12399,0),{complete}).

(bpe@127.0.0.1)5> bpe:history(39).
[#history{id = 28,version = undefined,container = feed,
          feed_id = {history,39},
          prev = 27,next = undefined,feeds = [],guard = true,
          etc = undefined,name = "Order11",
          task = {task,"end"}},
 #history{id = 27,version = undefined,container = feed,
          feed_id = {history,39},
          prev = 26,next = 28,feeds = [],guard = true,etc = undefined,
          name = "Order11",
          task = {task,"end2"}},
 #history{id = 26,version = undefined,container = feed,
          feed_id = {history,39},
          prev = undefined,next = 27,feeds = [],guard = true,
          etc = undefined,name = "Order11",
          task = {task,"begin"}}]
\end{lstlisting}

